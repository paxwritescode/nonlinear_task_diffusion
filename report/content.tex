\section{Постановка задачи}
\subsection{Математическая формулировка}
Найти функцию $u \in V$, где
\begin{equation}
\label{eq:V}
V := \left\{ w \in H^1(\Omega) :\; w\big|_{\Gamma_D} = 0 \right\},
\end{equation}
минимизирующую функционал
\begin{equation}
\label{eq:J}
J(u) = \int_{\Omega} \left( \frac{1}{2} |\nabla u|^2 + \frac{\alpha}{p}|u|^p - fu \right)dx,
\end{equation}
где

\begin{itemize}
	\item $\Omega \subset \mathds{R}^2$ - выпуклая ограниченная область с липшицевой границей;
	\item $\Gamma_D$ и $\Gamma_N$ - две непересекающиеся части границы $\partial\Omega$;
	\item $p>1$ - показатель нелинейности;
	\item $\alpha>0$ - коэффициент нелинейной реакции;
	\item $f$ - внешнее воздействие (источник/сток вещества).
\end{itemize}

Таким образом, требуется найти
\[
u = \operatorname*{arg\,min}_{w\in V} J(w).
\]
\subsection{Физические интерпретации модели}
Рассматривается процесс распространения некоторого вещества в ограниченной области. На движение вещества внутри области влияют два эффекта: диффузия, стремящаяся выровнять концентрацию, и внутренняя реакция, скорость которой нелинейно зависит от концентрации вещества. На части границы область находится под фиксированным воздействием, что моделируется условием Дирихле. На другой части границы фиксированное значение не задаётся, и вещество может свободно обмениваться с внешней средой, что соответствует условию Неймана. Интерес представляет стационарное распределение концентрации, которое устанавливается в системе под действием диффузии, нелинейной реакции и внешних факторов.\\

Функция $u(x)$ интерпретируется как концентрация вещества в точке $x \in \Omega$. Первое слагаемое функционала
$$\frac{1}{2} \int_{\Omega}|\nabla u|^2 dx$$
соответствует диффузионному переносу и характеризует энергетические затраты, связанные с неоднородностью распределения концентрации. Минимизация данного слагаемого отражает стремление системы к выравниванию концентрации.

Нелинейное слагаемое
$$\frac{\alpha}{p} \int_{\Omega} |u|^p dx$$
описывает внутреннюю реакцию, скорость которой нелинейно зависит от концентрации вещества, при этом параметр $p > 1$ определяет характер нелинейности, а коэффициент $\alpha > 0$ - её интенсивность.

Последнее слагаемое
$$ -\int_{\Omega} fu dx$$
учитывает влияние внешних источников или стоков вещества.
\section{Корректность модели (по Адамару)}
Задача математической физики является корректно поставленной по Адамару при выполнении следующих условий:
\begin{enumerate}
	\item существование решения;
	\item единственность решения;
	\item устойчивость решения.
\end{enumerate}
\subsection{Существование решения}
Рассматривается задача минимизации функционала
$\hyperref[eq:J]{J}$ на пространстве $\hyperref[eq:V]{V}$.
Пространство $V$ является замкнутым как линейное подпространство пространства $H^1 (\Omega)$ и непустым, так как нулевая функция принадлежит $V$.

Докажем коэрцитивность функционала J, на V т.е. покажем, что
\[
||u||_V \rightarrow \infty \Rightarrow J(u) \rightarrow + \infty.
\]
	Для этого оценим по модулю линейное слагаемое функционала
\(-\int_{\Omega} fu\,dx\). \\
Так как $V \subset H^1(\Omega)$, положим
\[ \|u\|_V := \|u\|_{H^1(\Omega)} \simeq \|\nabla u\|_{L^2(\Omega)}. \]
По неравенству Пуанкаре в пространстве $V$ имеем
\[ \|u\|_{L^2(\Omega)} \le C_P \|\nabla u\|_{L^2(\Omega)}. \]

Применяя неравенство Коши--Буняковского--Шварца, получаем
\[
\left| \int_{\Omega} fu\,dx \right| \le \|f\|_{L^2(\Omega)}\,\|u\|_{L^2(\Omega)}
\le \|f\|_{L^2(\Omega)} C_P \|\nabla u\|_{L^2(\Omega)}.
\]

Далее используем неравенство Юнга
\[ ab \le \frac{\varepsilon}{2}a^2 + \frac{1}{2\varepsilon}b^2, \quad \forall \varepsilon>0, \]
откуда
\[
\|f\|_{L^2(\Omega)} C_P \|\nabla u\|_{L^2(\Omega)}
\le \frac{\varepsilon}{2}\|\nabla u\|_{L^2(\Omega)}^2
+ \frac{C_P^2}{2\varepsilon}\|f\|_{L^2(\Omega)}^2.
\]

Следовательно,
\begin{align*}
	J(u) &= \frac12\int_{\Omega}|\nabla u|^2\,dx
	+ \frac{\alpha}{p}\int_{\Omega}|u|^p\,dx
	- \int_{\Omega}fu\,dx \\
	&\ge
	\frac{1}{2}\,\|\nabla u\|_{L^2(\Omega)}^2
	+ \frac{\alpha}{p}\,\|u\|_{L^p(\Omega)}^p
	- \left(
	\frac{\varepsilon}{2}\,\|\nabla u\|_{L^2(\Omega)}^2
	+ \frac{C_P^2}{2\varepsilon}\,\|f\|_{L^2(\Omega)}^2
	\right) \\
	&\ge
	\frac{1-\varepsilon}{2}\|\nabla u\|_{L^2(\Omega)}^2
	+ \frac{\alpha}{p}\|u\|_{L^p(\Omega)}^p
	- \frac{C_P^2}{2\varepsilon}\|f\|_{L^2(\Omega)}^2.
\end{align*}

Выбирая $\varepsilon=\tfrac14$, получаем оценку
\[
J(u) \ge \frac{3}{8}\|\nabla u\|_{L^2(\Omega)}^2
+ \frac{\alpha}{p}\|u\|_{L^p(\Omega)}^p - \tilde C,
\]
где $\tilde C = \mathrm{const}$.
Таким образом, функционал $J$ является коэрцитивным на пространстве $V$.\\

Также функционал является слабо полунепрерывным снизу в $V$, так как отображение $u \mapsto \int_{\Omega} |\nabla u|^2 dx$ выпукло и слабо непрерывно снизу, отображение $u \mapsto \int_{\Omega} |u|^p dx$ выпукло и слабо полунепрерывно снизу при $p > 1$, а линейный член $u \mapsto \int_{\Omega} fu \: dx$ слабо непрерывен.

Из коэрцитивности и слабой полунепрерывности снизу функционала $J$ следует существование минимизирующей последовательности, имеющей подпоследовательность, которая слабо сходится в $V$ к некоторой функции $u \in V$. Данная функция и будет минимизатором $J$ на $V$.

\subsection{Единственность решения}
Единственность решения следует из строгой выпуклости функционала $J$ на пространстве $V$, которая следует из строгой выпуклости первых двух слагаемых и того, что третье слагаемое не влияет на выпуклость. Строго выпуклый функционал не может иметь более одного минимизатора.

\subsection{Устойчивость решения}
Покажем, что решение задачи устойчиво относительно возмущений правой части $f$, т.е. малое изменение входных данных задачи влечёт за собой малое изменение решения.

Рассмотрим две функции $f_1, f_2 \in L^2(\Omega)$ и соответствующие им решения
$u_1, u_2 \in V$, являющиеся минимизаторами функционала $J$.

Функции $u_1$ и $u_2$ удовлетворяют уравнению Эйлера-Лагранжа:
\[
\int_{\Omega} \nabla u_i \cdot \nabla v \, dx
+ \alpha \int_{\Omega} |u_i|^{p-2} u_i v \, dx
= \int_{\Omega} f_i v \, dx,
\quad \forall v \in V, \; i=1,2.
\]

Вычитая эти равенства и выбирая в качестве тестовой функции
$v = u_1 - u_2$, получаем
\[
\int_{\Omega} |\nabla (u_1 - u_2)|^2 \, dx
+ \alpha \int_{\Omega}
\left( |u_1|^{p-2}u_1 - |u_2|^{p-2}u_2 \right)(u_1 - u_2)\, dx
= \int_{\Omega} (f_1 - f_2)(u_1 - u_2)\, dx.
\]

Так как функция $s \mapsto |s|^{p-2}s$ является монотонной при $p>1$, то
\[
\int_{\Omega}
\left( |u_1|^{p-2}u_1 - |u_2|^{p-2}u_2 \right)(u_1 - u_2)\, dx \ge 0.
\]

Следовательно,
\[
\|\nabla (u_1 - u_2)\|_{L^2(\Omega)}^2
\le
\left| \int_{\Omega} (f_1 - f_2)(u_1 - u_2)\, dx \right|.
\]

Применяя неравенство Коши--Буняковского--Шварца и неравенство Пуанкаре,
получаем оценку
\[
\|u_1 - u_2\|_V
\le C \|f_1 - f_2\|_{L^2(\Omega)},
\]
где постоянная $C$ не зависит от $f_1$ и $f_2$.

Таким образом, решение задачи непрерывно зависит от правой части, и задача является устойчивой.

\section{Дискретизация задачи}
\subsection{Дискретная постановка задачи}
Пусть $\mathcal{T}_h$ - регулярная триангуляция области $\Omega$ с параметром $h > 0$. Обозначим через $\Omega_h$ многоугольную аппроксимацию $\Omega$ на основе разбиения $\mathcal{T}_h$. 
Пусть $V_h \subset V$ - конечное аппроксимирующее пространство, состоящее из кусочно-линейных непрерывных функций, обращающихся в ноль на $\Gamma_D$:
\begin{equation}
	\label{eq:V_h}
	V_h = \{v_h \in C(\overline{\Omega}_h) : v_h |_K \in \mathds{P}_1(K) \: \forall K \in \mathcal{T}_h, \; v_h |_{\Gamma_D} = 0\}.
\end{equation}
Будем искать функцию $u_h \in V_h$, минимизирующую функционал
\begin{equation}
	\label{eq:J_h}
	J_h(v_h) = \int_{\Omega_h} \left( \frac{1}{2} |\nabla v_h|^2 + \frac{\alpha}{p} |v_h|^p - fv_h \right) dx, \quad v_h \in V_h.
\end{equation}
\subsection{Корректность дискретной постановки}
Функционал $J_h$ является непрерывным и коэрцитивным на $V_h$, следовательно, решение дискретной задачи существует. Его единственность следует из строгой выпуклости $J_h$ на $V_h$. Устойчивость решения дискретной задачи по отношению к правой части $f$ следует из строгой выпуклости функционала и непрерывной зависимости минимума выпуклого функционала от параметров.
\subsection{Сходимость к точному решению}
Пусть $u \in \hyperref[eq:V]{V}$ - точное решение исходной задачи минимизации функционала $\hyperref[eq:J]{J}$ на пространстве $V$, а $u_h \in \hyperref[eq:V_h]{\texorpdfstring{V_h}{V_h}}$ - решение дискретной задачи, т.е. минимизатор функционала $\hyperref[eq:V_h]{\texorpdfstring{J_h}{J_h}}$ на конечномерном пространчтве $V_h \in V$. 

Докажем, что при $h \rightarrow 0$ дискретное решение $u_h$ сходится к точному решению $u$ в пространстве $V$.\\

Так как $V_h$ состоит из кусочно-линейных функций на регулярной триангуляции и $V_h \subset V, \; \forall v \in V$ существует аппроксимирующая последовательность $v_h \in V_h$ такая, что
\[
||v - v_h||_V \rightarrow 0 \; \text{при} \; h \rightarrow 0.
\]
В частности, для точного решения $u \in V$ существует последовательность $u^*h \in V_h$. для которой
\[
||u - u_h^*||_V \rightarrow 0 \; \text{при} \; h \rightarrow 0.
\]
По определению дискретного решения $u_h$
\[
J(u_h) \leq J(v_h) \; \forall v_h \in V_h,
\]
в частности, для аппроксимирующей функции $u_h^* \in V_h$
\[
J(u_h) \leq J(u_h^*).
\]
Перейдём к пределу при $h \rightarrow 0$ и используем непрерывность функционала $J$ на $V$, получим
\[
\lim_{h \rightarrow 0} \sup J(u_h) \leq J(u),
\]
с другой стороны, из слабой полунепрерывности снизу функционала следует
\[
J(u) \leq \lim_{h \rightarrow 0} \inf J(u_h). 
\]
Отсюда
\[
\lim_{h \rightarrow 0} J(u_h) = J(u).
\]

Так как функционал $J$ строго выпуклый, из сходимости значений функционала следует сильная сходимость $u_h \rightarrow u$ в пространстве $V$.
\section{Решение конечномерной задачи}
Для того, чтобы свести исходную бесконечномерную задачу минимизации функционала к конечномерной задаче минимизации функции нескольких переменных, необходимо применить метод конечных элементов.
\subsection{Метод конечных элементов}
В точке минимума функционала первая вариация равна нулю во всех допустимых направлениях. Введём направление вариации $v_h$ как произвольную функцию из того же пространства $V_h$, потребуем, чтобы первая вариация функционала в точке $u_h$ обращалась в ноль по всем направлениям $v_h$. Таким образом, исходная конечномерная задача \eqref{eq:V_h}, \eqref{eq:J_h} сведётся к следующей:
\begin{equation}
\label{eq:after_first_der}
\int_{\Omega_h} \nabla u_h \cdot \nabla v_h dx + \alpha \int_{\Omega_h} |u_h|^{p - 2} u_h v_h dx = \int_{\Omega_h} fv_h dx, \quad \forall v_h \in V_h.
\end{equation}
\subsection{Метод Ньютона}
Для численного решения нелинейной задачи \eqref{eq:after_first_der} используется метод Ньютона.
После выбора базиса $\{\varphi_i\}_{i=1}^N$ пространства $V_h$ искомое решение
представляется в виде
\[
u_h = \sum_{i=1}^N U_i \varphi_i,
\]
где $U = (U_1, \dots, U_N)^T$ — вектор неизвестных коэффициентов.

Подставляя это представление в \eqref{eq:after_first_der} и выбирая в качестве тестовых
функций базисные функции $\varphi_j$, получаем систему нелинейных алгебраических уравнений
относительно вектора $U$.

Метод Ньютона заключается в построении последовательности приближений
$\{U^{(k)}\}$, сходящейся к решению этой системы. На каждом шаге метода
нелинейная система линеаризуется в окрестности текущего приближения
$U^{(k)}$, после чего решается соответствующая линейная система уравнений
для поправки $\delta U^{(k)}$.
Следующее приближение определяется по формуле
\[
U^{(k+1)} = U^{(k)} + \delta U^{(k)}.
\]

В качестве начального приближения $U^{(0)}$ может быть выбрано нулевое решение.
\section{Численные результаты и их анализ}
\section{Выводы}
\section{Список литературы}